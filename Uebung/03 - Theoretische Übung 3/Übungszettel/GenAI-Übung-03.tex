%{
% \documentclass[12pt,ngerman]{/Users/dominik-cau/Documents/Lernen/Uni/Promotion/Vorlagen/Bayreuth/Exercise/AssignmentClass}
\documentclass[12pt,ngerman]{AssignmentClass}
% \documentclass{article}
%\documentclass[12pt, english]{AssignmentClass}


%----------------------------------------------------------------------------------------
%	PACKAGES AND OTHER DOCUMENT CONFIGURATIONS
%----------------------------------------------------------------------------------------
% Template-specific packages
\usepackage[utf8]{inputenc} % Required for inputting international characters
\usepackage[T1]{fontenc} % Output font encoding for international characters
\usepackage{mathpazo} % Use the Palatino font
\usepackage{wasysym} % for flash-symbol
\usepackage{graphicx} % Required for including images
\usepackage{amsmath}
\usepackage{listings} % Required for insertion of code
\usepackage{siunitx}
\usepackage{pnets}
\usepackage[most]{tcolorbox} % Grey Deadline Bar
\DeclareMathAlphabet{\mathpzc}{OT1}{pzc}{m}{it}
% Initialize comment sections
\usetheme{light-theme}
\excludecomment{dark-theme}
\excludecomment{solution}


%----------------------------------------------------------------------------------------
%	SET VERSION
%----------------------------------------------------------------------------------------
%\includecomment{dark-theme}
\includecomment{solution}


%----------------------------------------------------------------------------------------
%	ASSIGNMENT INFORMATION
%----------------------------------------------------------------------------------------
%\setlanguageEnglish
%\setlanguageGerman
% \begin{dark-theme}
% 	\usetheme{dark-theme}
% 	\excludecomment{light-theme}
% \end{dark-theme}
\title{Übung 3} % Assignment title
\instructor{Yorck Zisgen}
\class{Generative Künstliche Intelligenz} % Course or class name
\term{Sommersemester 2024}
% \topics{Begriffe $\bullet$ Kontrollflussmuster $\bullet$ Organisationseinheiten}
\topics{28.05.2024}
%----------------------------------------------------------------------------------------
%}

\begin{document}
	\maketitle

    % Header Deadline Bar
    %{
    \noindent % Ensures the box spans the entire width
    \begin{tcolorbox}[colback=gray!20, % Background color as light gray
                      colframe=gray!20, % Frame color same as background
                      boxrule=0pt, % No border
                      sharp corners, % Sharp corners
                      valign=center, % Vertically centered text
                      halign=center, % Horizontally centered text
                      height=2cm] % Height of the box
    \LARGE \bfseries Abgabe: 09.06.2024 % Bold text
    \end{tcolorbox}
    %}
	
    \section{Foundation Models}
        Stellen Sie sich ein Einzelhandelsunternehmen vor, das ein Foundation Model wie DALL-E verwendet, um maßgeschneiderte Marketingmaterialien auf Basis von Verbraucherverhaltensdaten zu erstellen. Analysieren Sie, wie die Verwendung dieses KI-Modells die Marketingstrategien des Unternehmens transformieren kann. Behandeln Sie dabei die folgenden Aspekte:

        \begin{enumerate}[a)]
    		\item Erklären Sie, wie Foundation Models angepasst und eingesetzt werden können, um Marketingmaterialien an die individuellen Vorlieben der Verbraucher anzupassen.
    		\item Diskutieren Sie kurz das Potenzial für eine erhöhte Verbraucherbindung und Umsatzsteigerungen.
    		\item Gehen Sie auf mögliche Bedenken hinsichtlich Vollständigkeit, Kontext, Beziehung und Voreingenommenheit in multimodalen Daten ein.
    	\end{enumerate}

    
    \section{Herausforderungen im Umgang mit GenAI}
        \begin{enumerate}[a)]
            \item Finden Sie ein Beispiel für einen Model Bias, erklären Sie diesen und erläutern Sie mögliche Ursachen.
            \item Welche Folgen können Bias haben? Bringen Sie zwei Beispiele.
            \item Erläutern Sie das Phänomen „Halluzination“ von LLMs. Erklären Sie wirksame Strategien, diesem zu begegnen. 
        \end{enumerate}

    
    \section{Deep Fakes}
        Wie können Sie sich vor Deep Fakes schützen? Entwickeln Sie eine Strategie, wie Sie Deep Fakes besser erkennen können.

    
    \section{Rechtliche Aspekte}
        Was sind die wichtigsten Fakten zum EU AI Act (Stand der Gesetzgebung, Ziele des Gesetzes, Herangehensweise, Besonderheit)? 

 
\end{document}
