%{
% \documentclass[12pt,ngerman]{/Users/dominik-cau/Documents/Lernen/Uni/Promotion/Vorlagen/Bayreuth/Exercise/AssignmentClass}
\documentclass[12pt,ngerman]{AssignmentClass}
% \documentclass{article}
%\documentclass[12pt, english]{AssignmentClass}


%----------------------------------------------------------------------------------------
%	PACKAGES AND OTHER DOCUMENT CONFIGURATIONS
%----------------------------------------------------------------------------------------
% Template-specific packages
\usepackage[utf8]{inputenc} % Required for inputting international characters
\usepackage[T1]{fontenc} % Output font encoding for international characters
\usepackage{mathpazo} % Use the Palatino font
\usepackage{wasysym} % for flash-symbol
\usepackage{graphicx} % Required for including images
\usepackage{amsmath}
\usepackage{listings} % Required for insertion of code
\usepackage{siunitx}
\usepackage{pnets}
\usepackage[most]{tcolorbox} % Grey Deadline Bar
\DeclareMathAlphabet{\mathpzc}{OT1}{pzc}{m}{it}
% Initialize comment sections
\usetheme{light-theme}
\excludecomment{dark-theme}
\excludecomment{solution}
\hypersetup{urlcolor=cyan, colorlinks=true}
\usepackage{tabularx}
%----------------------------------------------------------------------------------------
%	SET VERSION
%----------------------------------------------------------------------------------------
%\includecomment{dark-theme}
\includecomment{solution}


%----------------------------------------------------------------------------------------
%	ASSIGNMENT INFORMATION
%----------------------------------------------------------------------------------------
%\setlanguageEnglish
%\setlanguageGerman
% \begin{dark-theme}
% 	\usetheme{dark-theme}
% 	\excludecomment{light-theme}
% \end{dark-theme}
\title{Praktische Übung 3} % Assignment title
\instructor{Kaan Apaydin

Kontakt für Fragen und Abgaben kaan.apaydin@uni-bayreuth.de}
\class{Generative Künstliche Intelligenz} % Course or class name
\term{Sommersemester 2024}
% \topics{Begriffe $\bullet$ Kontrollflussmuster $\bullet$ Organisationseinheiten}
\topics{08.07.2024}
%----------------------------------------------------------------------------------------
%}

\begin{document}
	\maketitle

    % Header Deadline Bar
    %{
    \noindent % Ensures the box spans the entire width
    \begin{tcolorbox}[colback=gray!20, % Background color as light gray
                      colframe=gray!20, % Frame color same as background
                      boxrule=0pt, % No border
                      sharp corners, % Sharp corners
                      valign=center, % Vertically centered text
                      halign=center, % Horizontally centered text
                      height=2cm] % Height of the box
    \LARGE \bfseries Abgabe: 21.07.2024 % Bold text
    \end{tcolorbox}
    %}

    
\section{Klassifikation von Informationen}
    
Für Begeisterte von generativer KI gibt es vielfältige Möglichkeiten eigene Projekte zu realisieren. LLMs könnten zum Beispiel zur automatisierten Klassifikation von Informationen eingesetzt werden. 
In der folgenden Aufgabe sollen Emails klassifiziert werden, um einer Kategorie zugeordnet zu werden und anschließend eine Antwort vorzuformulieren. 

        
\begin{enumerate}
\item Öffne das \href{https://colab.research.google.com/drive/1UCRWkXX_R3sopfzp_KzXBdkdrzukWwzE\#scrollTo=_x418HsXt1q-}{Python Notebook auf Google Colab} 
\item Erstelle eine einfache Skizze, die den Ablauf der Anwendung inklusive der Ein- und Ausgaben der Komponenten zeigt.
\item Analysiere den Code und beantworte die folgenden Fragen mit einer kurzen Erläuterung:
\begin{enumerate}
    \item Erzielt die Anwendung gute Ergebnisse entsprechend dem Anwendungsfall?
    \item Was passiert, wenn Sie die Reihenfolge der Ketten in der SequentialChain geändert werden?
    \item Welche Prompting-Techniken werden eingesetzt? 
\end{enumerate}
\item  Erstelle ein viertes Beispiel (z.B. mit ChatGPT) für einen Typ, der im Prompt nicht definiert wurde. Was passiert, wenn der Typ der E-Mail nicht in die definierten Kategorien fällt?
\item  Nehme die folgenden Änderungen am Code vor:
\begin{enumerate}
    \item Ändere die *classification\_prompt*, um unbekannte E-Mail-Typen explizit zu behandeln.
    \item Fügen der *classification\_prompt* eine weitere E-Mail-Kategorie hinzu, die ebenfalls über die *email\_answer\_chain* beantwortet werden soll.
    \item Fügen eine weitere Kette hinzu, die aus der Zusammenfassung eine aussagekräftige Überschrift mit bis zu 10 Wörtern erzeugt.
\end{enumerate}
\item (Optional) Werde kreativ und erstelle eine eigene SequentialChain mit mindestens 3 Kettengliedern.
\end{enumerate}

\textbf{Abgaben}
\begin{enumerate}[a)]
\item Skizze aus $Schritt\ 2$ 								
\item Erläuterungen zu den Aufgaben in $Schritt\ 3$
\item Dokumentation des vierten Beispiels und Beobachtung in $Schritt\ 4$
\item  Dokumentation der Änderungen in $Schritt\ 5$
\item (Optional) Dokumentation der eigenen SequentialChain aus $Schritt\ 6$
\end{enumerate}
\end{document}