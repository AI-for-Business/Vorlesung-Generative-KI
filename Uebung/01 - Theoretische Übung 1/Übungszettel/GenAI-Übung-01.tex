%{
% \documentclass[12pt,ngerman]{/Users/dominik-cau/Documents/Lernen/Uni/Promotion/Vorlagen/Bayreuth/Exercise/AssignmentClass}
\documentclass[12pt,ngerman]{AssignmentClass}
% \documentclass{article}
%\documentclass[12pt, english]{AssignmentClass}


%----------------------------------------------------------------------------------------
%	PACKAGES AND OTHER DOCUMENT CONFIGURATIONS
%----------------------------------------------------------------------------------------
% Template-specific packages
\usepackage[utf8]{inputenc} % Required for inputting international characters
\usepackage[T1]{fontenc} % Output font encoding for international characters
\usepackage{mathpazo} % Use the Palatino font
\usepackage{wasysym} % for flash-symbol
\usepackage{graphicx} % Required for including images
\usepackage{amsmath}
\usepackage{listings} % Required for insertion of code
\usepackage{siunitx}
\usepackage{pnets}
\usepackage[most]{tcolorbox} % Grey Deadline Bar
\DeclareMathAlphabet{\mathpzc}{OT1}{pzc}{m}{it}
% Initialize comment sections
\usetheme{light-theme}
\excludecomment{dark-theme}
\excludecomment{solution}


%----------------------------------------------------------------------------------------
%	SET VERSION
%----------------------------------------------------------------------------------------
%\includecomment{dark-theme}
\includecomment{solution}


%----------------------------------------------------------------------------------------
%	ASSIGNMENT INFORMATION
%----------------------------------------------------------------------------------------
%\setlanguageEnglish
%\setlanguageGerman
% \begin{dark-theme}
% 	\usetheme{dark-theme}
% 	\excludecomment{light-theme}
% \end{dark-theme}
\title{Übung 1} % Assignment title
\instructor{Yorck Zisgen}
\class{Generative Künstliche Intelligenz} % Course or class name
\term{Sommersemester 2024}
% \topics{Begriffe $\bullet$ Kontrollflussmuster $\bullet$ Organisationseinheiten}
\topics{23.04.2024}
%----------------------------------------------------------------------------------------
%}

\begin{document}
	\maketitle

    % Header Deadline Bar
    %{
    \noindent % Ensures the box spans the entire width
    \begin{tcolorbox}[colback=gray!20, % Background color as light gray
                      colframe=gray!20, % Frame color same as background
                      boxrule=0pt, % No border
                      sharp corners, % Sharp corners
                      valign=center, % Vertically centered text
                      halign=center, % Horizontally centered text
                      height=2cm] % Height of the box
    \LARGE \bfseries Abgabe: 06.05.2024 % Bold text
    \end{tcolorbox}
    %}

    
    \section{Modellierung vs. Machine Learning}
        Erklären Sie den Unterschied zwischen klassischer Modellierung und Machine Learning Ansätzen. Geben Sie jeweils ein Beispiel.
        
        % \textit{Die Lösung hierzu würde man auf Satz 1, Folie 18 finden.}


    
    
    \section{Grenzen von LLMs}
        
        \begin{enumerate}[a)]
        \item Diskutieren Sie die aktuellen Grenzen von LLMs. Welche Möglichkeiten sehen Sie, diese zu beheben?
        
        % \textit{Die Lösung hierzu würde man auf Satz 1, Folie 27 finden.}

        % \item Spielen Sie mit den Grenzen von ChatGPT, beispielsweise Wörter invertieren, Galgenraten, 4 Gewinnt, TicTacToe. Was ist Ihre Erkenntnis?
        \item Testen Sie folgendes mit ChatGPT: Wörter invertieren, Galgenraten, 4 Gewinnt, TicTacToe spielen. Fassen Sie kurz die Ergebnisse Ihrer Evaluation zusammen.
        \end{enumerate}


    
    
    \section{Praktische Anwendung}
        % \textit{Hier gibt es kein richtig und kein falsch, es geht ums 'selbst erleben'.}
    
        \begin{enumerate}[a)]
    		\item Lassen Sie von ChatGPT eine Modulbeschreibung zu einer Vorlesung schreiben, die „Einführung in die Generative KI“ heißt. Die Vorlesung ist für Masterstudierende der Wirtschaftswissenschaften gedacht.
    		\item Lassen Sie sich eine Übungsaufgabe zu der Vorlesung erstellen.
    		\item Erstellen Sie 5 Quizaufgaben zu der Vorlesung.
            \item Suchen Sie nach einem Beispiel, bei dem ChatGPT halluziniert.
    	\end{enumerate}

    
    \newpage
    
    \section{Kombinierte Anwendung}
        % \textit{Hier gibt es kein richtig und kein falsch, es geht ums 'selbst erleben'.}\\
    
        Generative KI kann nicht nur Texte erstellen.
        
        \begin{enumerate}[a)]
            \item Wählen sie ein Aufgabengebiet (Text-to-X oder X-to-Text). Recherchieren sie hierzu drei mögliche Produkte. Ermitteln Sie, ob sich manche davon kostenlos nutzen lassen.
            % \item Denken Sie sich ein Produkt aus. Lassen Sie sich eine Produktbeschreibung generieren. Lassen Sie sich ein Bild generieren. Erzeugen sie eine Homepage, auf der Produktname, -Bild und -Beschreibung dargestellt werden.
            \item Denken Sie sich ein Produkt aus und erzeugen Sie zu diesem Produkt eine Produktbeschreibung und ein fiktives Bild.

            \item Erzeugen sie eine Homepage, auf der Produktname, -Bild und -Beschreibung dargestellt werden. [\textit{Hinweis: ChatGPT-3.5 kann keine Bilder erzeugen. Sie können ChatGPT-4.0 verwenden wenn Sie Zugriff darauf haben, eine Alternative wie bspw. https://deepai.org benutzen oder etwas anderes wie einen Werbetext oder ein Video erzeugen.}]
        \end{enumerate}

        % \textit{Die Studierenden sollen sich mit unterschiedlichen Modi der GenAI auseinander setzen und "damit rumspielen". Eventuell ist da auch etwas bei, was sie weiterhin für das eigene Studium verwenden können?}

        % ** Anmerkung: [mit ChatGPT 3.5 wird das Erzeugen einer Seite mit diesen Inhalten schwierig. Vielleicht Aufgabe weglassen oder was anderes machen lassen] Erzeugen sie eine Homepage, auf der Produktname, -Bild und -Beschreibung dargestellt werden.

    
\end{document}

